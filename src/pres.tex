\documentclass{beamer}
\usetheme{ucsandiego} 

\usepackage{color} 
\usepackage{graphicx} % external images
\usepackage[font=small]{caption} % caption numbers
\usepackage{tikz}

\theoremstyle{definition}




\title{Improving Low-Resource Neural~Machine~Translation}
\author{Leon Cheung} 
\institute{University of California San Diego}
\date{8 September 2017} % custom date

% Comment this section out to disable table of contents at section begin.
\AtBeginSection[]
{
\begin{frame}
\frametitle{Contents}
% This will display the table of contents and highlight the current section.
\tableofcontents[currentsection] 
\end{frame}
}


\begin{document}

{
\usebackgroundtemplate{
    \tikz[overlay,remember picture]
    \node[opacity=0.05, at=(current page.south east),anchor=south east,inner sep=0pt] {
        \includegraphics[height=\paperheight,width=\paperwidth]{imgs/trident-30-4x3-right-bot}};
}
\begin{frame} 
\titlepage
\end{frame}
}

\section{What Can Happen at a Critical Point?} 

\begin{frame} 
\frametitle{The Usual Suspects}

\begin{center}
\begin{tabular}{l | r}
    \hline
    jsdf & 0.23 \\
    aaaa & 0.54
\end{tabular}
\end{center}
\graybox{sdf}{sdffff}

You might think that if $f'(0)=0$ (and $f$ is not a constant function) then at 
$x=0$, $f$ must have
\begin{itemize}
\pause \item a local maximum, or 
\pause \item a local minimum, or
\end{itemize}  
\end{frame}

\begin{frame}
\frametitle{A Counterexample}
\alert{Consider} the function 
\[ 
    f(x)=
\begin{cases} 
x^2\sin(1/x), &\text{if }x\neq0 \\
0, &\text{if }x=0
\end{cases}
\]
\end{frame}

\begin{frame}
\frametitle{What Really Happens at $x=0$?}
\begin{columns} 
\begin{column}{0.5\textwidth}
But $f(x)$ oscillates wildly as $x\to 0$, so even though $f'(0)=0$, $f$ has 
neither max, min, nor inflection point at $x=0$.
\end{column}
\pause
\begin{column}{0.5\textwidth}
%    \includegraphics[width=5cm, height=5cm]{graph1.png}
    \captionof{figure}{caption}
\end{column}
\end{columns}
\end{frame}

\subsection{What Does $g'(c)>0$ Mean?} 
\begin{frame}
\frametitle{What Really Happens at $x=0$?}

\begin{center}
%\includegraphics[width=5cm, height=5cm]{graph2.png}
\captionof{figure}{caption}
\end{center}

\end{frame}

\section{Conclusion}

{
\usebackgroundtemplate{
    \tikz[overlay,remember picture]
    \node[opacity=0.05, at=(current page.south east),anchor=south east,inner sep=0pt] {
        \includegraphics[height=\paperheight,width=\paperwidth]{imgs/trident-30-4x3-right-bot}};
}
\begin{frame}
The function $f(x)$ introduced earlier has other interesting properties, 
one of which is the fact that while $f'(0)$ exists, $f'(x)$ is 
discontinuous at $x=0$.
\vspace{.5cm}

We leave it to you to work this out for yourself and to explore this 
interesting function further.
\vspace{.5cm}

Thank you for your attention today.
\end{frame}
}

\end{document}

